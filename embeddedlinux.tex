\documentclass[10pt,compress]{beamer}
%\documentclass[fleqn,xcolor=dvipsnames]{beamer}
\usetheme{Madrid}
%\usetheme{Boadilla}
%\usetheme{default}
%\usetheme{Warsaw}
%\usetheme{CambridgeUS}
%\usetheme{Sybila}
%\usetheme[hideothersubsections]{Berkeley}
%\usetheme[hideothersubsections]{PaloAlto}
%%\usetheme[hideothersubsections]{Goettingen}
%\usetheme{CambridgeUS}
%\usetheme{Bergen} % This template has nagivation on the left
%\usetheme{Frankfurt} % Similar to the default with an extra region at the top.

% Uncomment the following line if you want page numbers and using Warsaw theme%
%\setbeamertemplate{footline}[page number]

\definecolor{aggiemaroon}{RGB}{80,0,0} % Official RGB code for aggie maroon
\usecolortheme[named=aggiemaroon]{structure}
\useoutertheme{shadow}
\useinnertheme{rounded}
\setbeamertemplate{navigation symbols}{}
\setbeamerfont{structure}{family=\rmfamily,series=\bfseries}
\setbeamerfont{footline}{size=\tiny}
\usefonttheme[stillsansseriftext]{serif}

\usepackage{amsmath}
\usepackage{caption}
\usepackage{graphicx,comment}
\usepackage[english]{babel}
\usepackage{rotating}
\usepackage{multicol}
\usepackage{enumerate}
\usepackage{tikz}
\usepackage{bm}
\usepackage{listings}
\usepackage{minted}
\usepackage{tabularx}
\usepackage{hyperref} %always the last to avoid problems


\usebackgroundtemplate{
\tikz\node[opacity=0.035, rotate = 0] {\includegraphics[scale=0.5]{UPM-logo.png}};}
%{\includegraphics[height=1in,width=1in]{TAM-Logo.png}};}


\title[Universidad Politecnica de Madrid]{Embedded System Design with Raspberry-pi}

\author [M. Ruiz]{Mariano Ruiz}

\date[01/15/2025]{January 15, 2025}

\institute[UPM] % (optional, but mostly needed)
{\emph{Professor}, ETSI Sistemas de Telecomunicación Copiado de Bootlin \\
 \includegraphics[scale=0.60]{UPM-logo.png} \\ [0.0 cm]
 }



\AtBeginSection[]
{
  \begin{frame}<beamer>
   % \frametitle{Section \thesection} %this add the section number only
   \tiny
   \frametitle{\insertsectionhead}
    \tableofcontents[currentsection, hideallsubsections]
  \end{frame}
}
%\newcommand{\codelink}[1]{\texttt{#1}}

\newcommand{\codecolor}{\usebeamercolor[fg]{code}}
\newcommand{\code}[1]{{\codecolor \path{#1}}}
\newcommand{\codelink}[1]{{\usebeamercolor[fg]{darkblue} \path{#1}}}
\newcommand{\codewithhash}[1]{{\codecolor \tt{#1}}}
%\newcommand{\code}[1] {\path{#1}}

\newcommand\projdir[2]{\href{https://elixir.bootlin.com/#1/latest/source/#2/}{\codelink{#2/}}}
\newcommand\projfile[2]{\href{https://elixir.bootlin.com/#1/latest/source/#2}{\codelink{#2}}}
\newcommand\kconfig[1]{\href{https://elixir.bootlin.com/linux/latest/K/ident/#1}{\codelink{#1}}}
\newcommand\kconfigval[2]{\href{https://elixir.bootlin.com/linux/latest/K/ident/#1}{\codelink{#1=#2}}}
\newcommand\kdir[1]{\projdir{linux}{#1}}
\newcommand\kfile[1]{\projfile{linux}{#1}}
\newcommand\kfileversion[2]{\href{https://elixir.bootlin.com/linux/v#2/source/#1}{\codelink{#1}}}
\newcommand\kstruct[1]{\href{https://elixir.bootlin.com/linux/latest/ident/#1}{\codelink{struct #1}}}
\newcommand\kdochtml[1]{\href{https://www.kernel.org/doc/html/latest/#1.html}{\codelink{#1}}}
\newcommand\kdochtmldir[1]{\href{https://www.kernel.org/doc/html/latest/#1/}{\codelink{#1/}}}

\begin{document}

\begin{frame}
\titlepage
\end{frame}

%\begin{frame}
%\frametitle{Overview} 
%\tableofcontents
%\end{frame}

\newcommand{\training}{linux-kernel}
\graphicspath{{trainingmaterials/}}
\input{trainingmaterials/slides/sysdev-intro/sysdev-intro}
\input{trainingmaterials/slides/sysdev-dev-environment/sysdev-dev-environment}
\section[CROSS]{Cross-compiling toolchains}
\input{trainingmaterials/slides/sysdev-toolchains-definition/sysdev-toolchains-definition}
\input{trainingmaterials/slides/sysdev-toolchains-options/sysdev-toolchains-options}
\input{trainingmaterials/slides/sysdev-toolchains-obtaining/sysdev-toolchains-obtaining}
\section[BL]{Bootloaders and firmware}
\input{trainingmaterials/slides/sysdev-bootloaders-sequence/sysdev-bootloaders-sequence}
\section[Kernel]{Linux kernel introduction}

\input{trainingmaterials/slides/sysdev-linux-intro-features/sysdev-linux-intro-features}
\input{trainingmaterials/slides/sysdev-linux-intro-versioning/sysdev-linux-intro-versioning}
\input{trainingmaterials/slides/sysdev-linux-intro-sources/sysdev-linux-intro-sources}
\input{trainingmaterials/slides/sysdev-kernel-building/sysdev-kernel-building}
\input{trainingmaterials/slides/sysdev-kernel-booting/sysdev-kernel-booting}
\section[RFS]{Linux Root Filesystem}

\input{trainingmaterials/slides/sysdev-root-filesystem-principles/sysdev-root-filesystem-principles}
\input{trainingmaterials/slides/sysdev-root-filesystem-contents/sysdev-root-filesystem-contents}
\input{trainingmaterials/slides/sysdev-root-filesystem-virtual-fs/sysdev-root-filesystem-virtual-fs}
\input{trainingmaterials/slides/boot-sequence-initramfs/boot-sequence-initramfs}
\input{trainingmaterials/slides/sysdev-busybox/sysdev-busybox}
\input{trainingmaterials/slides/sysdev-hw-devices/sysdev-hw-devices}
\input{trainingmaterials/slides/buildroot-yocto-introduction/buildroot-yocto-introduction}
\input{trainingmaterials/slides/buildroot-introduction/buildroot-introduction}
\input{trainingmaterials/slides/buildroot-build/buildroot-build}
\input{trainingmaterials/slides/buildroot-tree/buildroot-tree}


%------------------------------------------------

\begin{frame}{}
  \centering \Huge
  \emph{Thank You!}
\end{frame}

%----------------------------------------------------------------------------------------


\end{document}

